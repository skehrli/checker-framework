\htmlhr
\chapterAndLabel{Resource Leak Checker for must-call obligations}{resource-leak-checker}

The Resource Leak Checker guarantees that the program fulfills every object's
must-call obligations before the object is de-allocated.

A resource leak occurs when a program does not explicitly dispose of some finite
underlying resource, such as a socket, file descriptor, or database connection.  To dispose
of the resource, the program should call some method on an object.
(De-allocating or garbage-collecting the object is not sufficient.)  For
example, the program must call \<close()> on every object that implements the
interface \<java.io.Closeable>.

The Resource Leak Checker can check any property of the form ``the programmer
must call each method in a set of methods \emph{M} at least once
on object \emph{O} before \emph{O} is de-allocated''.  For resource leaks,
by default \emph{M} is the set containing
\<close()> and \emph{O} is any object that implements the interface
\<java.io.Closeable>.  You can extend this guarantee to other types and methods
by writing \<@MustCall> or \<@InheritableMustCall> annotations, as described in
Section~\ref{must-call-annotations}.

The Resource Leak Checker works in three stages:
\begin{enumerate}
\item The Must Call Checker (\chapterpageref{must-call-checker})
  over-approximates each expression's must-call methods as a
  \refqualclass{checker/mustcall/qual}{MustCall} type.
\item The Called Methods Checker (\chapterpageref{called-methods-checker})
  under-approximates each expression's definitely-called methods as a
  \refqualclass{checker/calledmethods/qual}{CalledMethods} type.
\item When any program element goes out of scope (i.e., it is ready to be
  de-allocated), the Resource Leak Checker compares the types
  \<@MustCall(\emph{MC})> and \<@CalledMethods(\emph{CM})>.  It reports an error
  if there exists some method in \emph{MC} that is not in \emph{CM}.
\end{enumerate}


\sectionAndLabel{How to run the Resource Leak Checker}{resource-leak-run-checker}

Run one of these lines:

\begin{Verbatim}
javac -processor resourceleak MyFile.java ...
javac -processor org.checkerframework.checker.resourceleak.ResourceLeakChecker MyFile.java ...
\end{Verbatim}

The Resource Leak Checker supports all the command-line arguments
listed in Section~\ref{called-methods-run-checker} for
the Called Methods Checker, plus four others:

\begin{description}
\item[\<-ApermitStaticOwning>]
  See Section~\ref{resource-leak-owning-fields}.
\item[\<-AresourceLeakIgnoredExceptions=...>]
  See Section~\ref{resource-leak-ignored-exceptions}.
\item[\<-ApermitInitializationLeak>]
  See Section~\ref{resource-leak-field-initialization}.
\item[\<-AenableReturnsReceiverForRlc>]
  Enables the Returns Receiver Checker (Section~\ref{returns-reciever-checker})
  when running this checker. By default, the Returns Receiver checker is
  disabled when running the Called Methods Checker as part of the RLC,
  because it has a significant runtime cost but provides little benefit in the common case,
  since resources are rarely combined with fluent builders.
%% TODO: Uncomment when the feature is ready to be publicized.
% \item[\<-AenableWpiForRlc>]
%   See Section~\ref{resource-leak-checker-inference-algo}.
\end{description}

If you are running the Resource Leak Checker, then there is no need to run
the Must Call Checker (\chapterpageref{must-call-checker}), because the
Resource Leak Checker does so automatically.


\sectionAndLabel{Resource Leak Checker annotations}{resource-leak-annotations}

The Resource Leak Checker relies on the type qualifiers of two other checkers:
the Must Call Checker (Section~\ref{must-call-annotations}) and
the Called Methods Checker (Section~\ref{called-methods-spec}). You might need
to write qualifiers from either type hierarchy. The most common annotations from
these checkers that you might need to write are:

\begin{description}

\item[\refqualclasswithparams{checker/mustcall/qual}{MustCall}{String[] value}]
for example on an element with compile-time type \<Object> that might contain a \<Socket>.
See Section~\ref{must-call-annotations}.

\item[\refqualclasswithparams{checker/mustcall/qual}{InheritableMustCall}{String[] value}]
on any classes defined in your program that have must-call obligations. See Section~\ref{must-call-on-class}.

\item[\refqualclass{checker/calledmethods/qual}{EnsuresCalledMethods} and/or
      \refqualclass{checker/calledmethods/qual}{EnsuresCalledMethodsOnException}]
on a method in your code that fulfills a must-call obligation of one of its parameters or of a field.
See Section~\ref{called-methods-ensurescalledmethods}.

\end{description}

The Resource Leak Checker supports annotations that express
aliasing patterns related to resource leaks:

\begin{description}

\item[\refqualclass{checker/mustcall/qual}{Owning}]
\item[\refqualclass{checker/mustcall/qual}{NotOwning}]
  expresses ownership.  When two aliases exist to the same Java object,
  \<@Owning> and \<@NotOwning> indicate which of the two is responsible for
  fulfilling must-call obligations.
  Constructor results are always \<@Owning>. Method returns default to
  \<@Owning>.  Formal parameters and fields default to \<@NotOwning>.
  For more details, see Section~\ref{resource-leak-ownership}.

\item[\refqualclass{checker/mustcall/qual}{MustCallAlias}]
  represents a ``resource-aliasing'' relationship.  Resource aliases are
  distinct Java objects that control the same resource(s):
  fulfilling the must-call obligations of one object also
  fulfills the obligations of the other object.  For details,
  see Section~\ref{resource-leak-resource-alias}.

\end{description}

The Resource Leak Checker also supports an annotation to permit re-assigning
fields or re-opening resources:

\begin{description}

\item[\refqualclasswithparams{checker/mustcall/qual}{CreatesMustCallFor}{String value}]
  is a declaration annotation that indicates that after a call to a method
  with this annotation none of the must-call obligations of the in-scope, owning expression
  listed in \<value> have been met.
  In other words, the annotated method ``resets'' the must-call obligations of the expression.
  Multiple \<@CreatesMustCallFor>
  annotations can be written on the same method.  Section~\ref{resource-leak-createsmustcallfor}
  explains how this annotation permits re-assignment of owning
  fields or the re-opening of resources.

\end{description}


\subsectionAndLabel{Resource Leak Checker annotations for collections}{resource-leak-annotations-collections}

The Resource Leak Checker also supports collections of resources. The following annotations express ownership over resource collections.

\begin{description}
\item[\refqualclass{checker/collectionownership/qual}{OwningCollection}]
\item[\refqualclass{checker/collectionownership/qual}{NotOwningCollection}]
  expresses ownership over resource collections. Resource collection formal parameters default to \texttt{@NotOwningCollection}, while resource collection fields and return types default to \texttt{@OwningCollection}.
  For more details, see Section~\ref{resource-leak-collections}.

\end{description}

This annotation is for dedicated destructor methods of resource collection fields.

\begin{description}
\item[\refqualclasswithparams{checker/collectionownership/qual}{CollectionFieldDestructor}{String[] value}]
  expresses that the annotated method fulfills the obligations of the given resource collection fields.
  For more details, see Section~\ref{resource-leak-collections-fields}.

\end{description}


\sectionAndLabel{Example of how safe resource usage is verified}{resource-leak-example}

Consider the following example of safe use of a \<Socket>, in which the comments indicate the
inferred Must Call and Called Methods type qualifiers for \<s>:
\begin{verbatim}
{
  Socket s = null;
  // 1. @MustCall({}) @CalledMethodsBottom
  try {
    s = new Socket(myHost, myPort);
    // 2. @MustCall("close") @CalledMethods({})
  } catch (Exception e) {
    // do nothing
  } finally {
    if (s != null) {
      s.close();
      // 3. @MustCall("close") @CalledMethods("close")
    } else {
      // do nothing
      // 4. @MustCall("close") @CalledMethodsBottom
    }
    // 5. @MustCall("close") @CalledMethods("close")
  }
  // 6. @MustCall("close") @CalledMethods("close")
}
\end{verbatim}

At point 1, \<s>'s type qualifiers are the type qualifiers of \<null>:
\<null> has no must-call obligations (\<@MustCall(\{\})>),
and methods cannot be called on it (\<@CalledMethodsBottom>).

At point 2, \<s> is a new \<Socket> object, which
has a must-call obligation (\<@MustCall("close")>)
and has had no methods called on it (\<@CalledMethods(\{\})>).

At point 3, \<close()> has definitely been called on \<s>, so
\<s>'s Called Methods type is updated. Note that the Must Call type
does not change.

At point 4, \<s> is definitely \<null> and its type is adjusted accordingly.

At point 5, \<s>'s type is the least upper bound of the types at points 3
and 4.

At point 6, \<s> goes out of scope.  The Resource Leak Checker reports a
\<required.method.not.called> error if the Must Call set contains any
element that the Called Methods set does not.


\sectionAndLabel{Aliased references and ownership transfer}{resource-leak-ownership}

Resource leak checking is complicated by aliasing.  Multiple expressions
may evaluate to the same Java object, but each object only needs to be
closed once.  (Section~\ref{resource-leak-resource-alias} describes a
related situation called ``resource aliasing'', when multiple Java objects
refer to the same underlying resource.)

For example, consider the following code that safely closes a \<Socket>:

\begin{verbatim}
  void example(String myHost, int myPort) throws IOException {
    Socket s = new Socket(myHost, myPort);
    closeSocket(s);
  }
  void closeSocket(@Owning @MustCall("close") Socket t) {
    try {
      t.close();
    } catch (IOException e) {
      e.printStackTrace();
    }
  }
\end{verbatim}

There are two aliases for a socket object: \<s> in \<example()> and \<t> in
\<closeSocket()>.  Ordinarily, the Resource Leak Checker requires that
\<close()> is called on every expression of type \<Socket>, but that is not
necessary here.  The Resource Leak Checker should not warn when
\<s> goes out of scope in \<example()>, because \<closeSocket()> takes ownership
of the socket --- that is, \<closeSocket()> takes responsibility for closing
it. The \<@Owning> annotation on \<t>'s declaration expresses this fact; it
tells the Resource Leak Checker that \<t> is the reference that must be
closed, and its alias \<s> need not be closed.

Constructor returns are always \<@Owning>.
Method returns default to \<@Owning>,
and parameters and fields default to \<@NotOwning>. This treatment of parameter and
return types ensures sound handling of unannotated third-party libraries: any
object returned from such a library will be tracked by default, and the checker
never assumes that passing an object to an unannotated library will satisfy its obligations.

\<@Owning> and \<@NotOwning> always \emph{transfer} must-call obligations: must-call
obligations are conserved (i.e., neither created nor destroyed) by ownership annotations.
Writing \<@Owning> or \<@NotOwning> can never make the checker
unsound:  a real warning can never be hidden by them.
As with any annotation, incorrect or missing annotations can lead to false positive warnings.

\subsectionAndLabel{Owning parameters and exceptions}{resource-leak-owning-parameters-and-exceptions}

When \<@Owning> is written on a method parameter, the method only takes ownership of the
parameter when it returns normally.  In this example, the Resource Leak Checker will report
an error in the \<example> method and allow the definition of \<closeSocket>:

\begin{verbatim}
  void example(String myHost, int myPort) throws Exception {
    // Error: `s` is not closed on all paths
    Socket s = new Socket(myHost, myPort);

    // `closeSocket` does not have to close `s` when it throws IOException.
    // Instead, this method has to catch the exception and close `s`.
    closeSocket(s);
  }

  void closeSocket(@Owning Socket t) throws IOException {
    throw new IOException();
  }
\end{verbatim}

Sometimes a method really does promise to call some methods on an \<@Owning> parameter,
even if it throws an exception.  The annotation \<@EnsuresCalledMethodsOnException> can
overcome this limitation.  For example, a constructor that throws an exception might
choose to close an \<@Owning> parameter instead of letting ownership remain with the caller:

\begin{verbatim}
  @EnsuresCalledMethodsOnException(value = "#1", methods = "close")
  public Constructor(@Owning Closeable resource) {
    this.resource = resource;
    try {
      initialize();
    } catch (Exception e) {
      resource.close();
      throw e;
    }
  }
\end{verbatim}

\subsectionAndLabel{Owning fields}{resource-leak-owning-fields}

Unannotated fields are treated as non-owning.

For \textbf{final, non-static owning fields},
the Resource Leak Checker enforces the ``resource acquisition is
initialization (RAII)'' programming idiom.  Some
destructor-like method \<d()> must satisfy the field's must-call obligation
(and this fact must be expressed via a \<@EnsuresCalledMethods> annotation on \<d()>),
and the enclosing class must have a \<@MustCall("d")> obligation to
ensure the destructor is called. In addition to the \<@EnsuresCalledMethods> annotation,
which guarantees that the field(s) it references have their must-call obligations satisfied
on non-exceptional paths, the Resource Leak Checker requires those fields to have their must-call
obligations satisfied on all paths in (only) the destructor, and will issue a \<destructor.exceptional.postcondition>
error if they are not satisfied. Resolve this error by ensuring that the required methods are called
on all exceptional paths.

\textbf{Non-final, non-static owning fields} usually require one or more \<@CreatesMustCallFor> annotations
when they might be re-assigned. See Section~\ref{resource-leak-createsmustcallfor} for
more details on how to annotate a non-final, non-static owning field.

Owning fields are treated slightly differently in constructors versus normal methods.
In normal methods, assigning a value to an owning field always satisfies the object's
must-call obligations.  However, within a constructor, those obligations are only
satisfied if the constructor returns normally.  If the constructor throws an exception,
the constructed object will not be accessible afterward, and therefore its fields need
to be closed before it exits.

This constructor safely closes the object it allocates before throwing an exception:
\begin{verbatim}
  private final @Owning Socket socket;

  public ConstructorThatCanThrow() throws IOException {
    Socket s = new Socket(myHost, myPort);
    try {
      initialize(s); // may throw IOException
    } catch (Exception e) {
      s.close();
      throw e;
    }
    this.socket = s;
  }
\end{verbatim}

An assignment to a \textbf{static owning field} does not satisfy a
must-call obligation; for example,

\begin{smaller}
\begin{Verbatim}
  static @Owning PrintWriter debugLog = new PrintWriter("debug.log");
\end{Verbatim}
\end{smaller}

\noindent
The Resource Leak Checker issues a warning about every assignment of an
object with a must-call annotation into a static owning field,
indicating that the obligation of the field's content might not be
satisfied.  When those fields are used throughout execution, until the
program exits, there is no good place to dispose of them, so these warnings
might not be useful.  The \<-ApermitStaticOwning> command-line argument
suppresses warnings related to static owning fields.  This can help in
checking legacy code.  It permits only a small number of resource retained
throughout execution, related to the number of such fields and assignments
to them.


\sectionAndLabel{Resource aliasing}{resource-leak-resource-alias}

A \emph{resource alias} set is a set of Java objects that
correspond to the same underlying system resource.
Calling a must-call method on any member of a resource-alias set
fulfills that obligation for all members of the set.
Members of the set may have different Java types.

Programmers most often encounter resource aliasing when using \emph{wrapper types}.
For example, the Java \<Buffered\-Output\-Stream> wrapper adds buffering to a
delegate stream.
The wrapper's \<close()> method invokes \<close()> on the delegate.  Calling
\<close()> on either object has the same effect:  it closes the underlying resource.

A resource aliasing relationship is expressed in source code via a pair of \<@MustCallAlias> annotations:
one on a parameter of a method or constructor, and another on its return type.
For example, the annotated JDK contains this constructor of \<BufferedOutputStream>:
\begin{Verbatim}
@MustCallAlias BufferedOutputStream(@MustCallAlias OutputStream out);
\end{Verbatim}

When a pair of \<@MustCallAlias> annotations is written on a method or constructor \<m>'s return type
and its parameter \<p>, the Resource Leak Checker requires one of the following:
\begin{enumerate}
\item \<p> is passed to another method or constructor (including \<super>) in a
  \<@MustCallAlias> position, and \<m> returns that method's result, or
\item \<p> is stored in the only \<@Owning> field of the enclosing class (a class with more than one
  \<@Owning> field cannot have a resource alias relationship).
\end{enumerate}

\subsectionAndLabel{A complete wrapper type example}{resource-leak-wrapper-type-example}

Here is a complete example of a type \<InputStreamWrapper> that wraps an \<InputStream> as a resource alias.  Defining a wrapper type typically involves combined usage of \<@InheritableMustCall>, \<@EnsuresCalledMethods>, an \<@Owning> field, and \<@MustCallAlias>.  The \<test> method shows that the checker is able to verify code that releases an \<InputStream> using either the \<InputStream> directly or a wrapping \<InputStreamWrapper>.

\begin{verbatim}
@InheritableMustCall("dispose")
class InputStreamWrapper {
  private final @Owning InputStream stream;

  @MustCallAlias InputStreamWrapper(@MustCallAlias InputStream stream) {
    this.stream = stream;
  }

  @EnsuresCalledMethods(value = "this.stream", methods = "close")
  public void dispose() throws IOException {
    this.stream.close();
  }

  /** Shows that either the stream or the wrapper can be closed. */
  static void test(@Owning InputStream stream, boolean b) throws IOException {
    InputStreamWrapper wrapper = new InputStreamWrapper(stream);
    if (b) {
      stream.close();
    } else {
      wrapper.dispose();
    }
  }
}
\end{verbatim}


\sectionAndLabel{Creating obligations (how to re-assign a non-final owning field)}{resource-leak-createsmustcallfor}

Consider a class that has must-call obligations; that is, the class
declaration is annotated with \<@MustCall(...)>.
Every constructor implicitly creates obligations for the newly-created object.
Non-constructor methods may also create obligations
when re-assigning non-final owning fields or allocating
new system-level resources.

A post-condition annotation,
\<@CreatesMustCallFor>,
indicates for which expression an obligation is created.
If you write \<@CreatesMustCallFor(>\emph{T}\<)> on a method \emph{N} that
overrides a method \emph{M}, then \emph{M} must also be annotated as
\<@CreatesMustCallFor(>\emph{T}\<)>.  (\emph{M} may also have other
\<@CreatesMustCallFor> annotations that \emph{N} does not.)

\<@CreatesMustCallFor> allows the Resource Leak Checker to verify uses of non-final fields
that contain a resource, even if they are re-assigned. Consider
the following example:

\begin{verbatim}
  @MustCall("close") // default qualifier for uses of SocketContainer
  class SocketContainer {
    private @Owning Socket sock;

    public SocketContainer() { sock = ...; }

    void close() { sock.close() };

    @CreatesMustCallFor("this")
    void reconnect() {
      if (!sock.isClosed()) {
        sock.close();
      }
      sock = ...;
    }
  }
\end{verbatim}

In the lifetime of a \<SocketContainer> object, \<sock>
might be re-assigned arbitrarily many times: once at each
call to \<reconnect()>. This code is safe, however: \<reconnect()>
ensures that \<sock> is closed before re-assigning it.

Sections~\ref{resource-leak-createsmustcallfor-callsite}
and~\ref{resource-leak-createsmustcallfor-declaration}
explain how the Resource Leak Checker verifies uses and declarations of
methods annotated with \<@CreatesMustCallFor>.


\subsectionAndLabel{Requirements at a call site of a \<@CreatesMustCallFor> method}{resource-leak-createsmustcallfor-callsite}

At a call site to a method annotated as
\<@CreatesMustCallFor(>\emph{expr}\<)>, the Resource Leak Checker:
\begin{enumerate}
\item
  Treats any existing \<@MustCall> obligations of \emph{expr} as \emph{satisfied},
\item
  Creates a fresh obligation to check, as if \emph{expr} was assigned to a newly-allocated
  object (i.e. as if \emph{expr} were a constructor result).
\item
  Un-refines the type in the Called Methods Checker's type hierarchy for \emph{expr} to
  \<@CalledMethods(\{\})>, if it had any other Called Methods type.
\item
  Requires that the expression corresponding to \emph{expr} (that is, \emph{expr}
  viewpoint-adapted to the method call site) is owned; that is, it is
  annotated or defaulted as \<@Owning>.  Otherwise, the checker
  will issue a \<reset.not.owning> error at the call-site. You can avoid this
  error by extracting \emph{expr} into a new local variable (because
  locals are \<@Owning> by default) and replacing all instances of \emph{expr}
  in the call with references to the new local variable.
\end{enumerate}

Treating the obligation before the call as satisfied is sound: the
checker creates a new obligation for calls to \<@CreatesMustCallFor> methods,
and the Must Call Checker (\chapterpageref{must-call-checker}) ensures the
\<@MustCall> type for the target expression will have a \emph{superset} of any methods
present before the call. Intuitively, calling an \<@CreatesMustCallFor> method
``resets'' the obligations of the target expression, so whether they were satisfied before
the call or not is irrelevant.

If an \<@CreatesMustCallFor>
method \emph{n} is invoked within a method \emph{m} that has an \<@CreatesMustCallFor> annotation,
and the \<@CreatesMustCallFor> annotations on \emph{n} and \emph{m} have
the same target expression---imposing the obligation produced by calling \emph{n} on the caller of \emph{m}---then
the newly-created obligation is treated as satisfied immediately
at the call-site of \emph{n} in the body of \emph{m} (because it is imposed at call-sites of \emph{m}
instead).


\subsectionAndLabel{Requirements at a declaration of a \<@CreatesMustCallFor> method}{resource-leak-createsmustcallfor-declaration}

Any method that re-assigns a non-final, owning field of some object \<obj>
must be annotated \<@CreatesMustCallFor("obj")>.
Other methods may also be annotated with \<@CreatesMustCallFor>.

The Resource Leak Checker enforces two rules to ensure that re-assignments
to non-final, owning fields (like \<sock> in method \<reconnect> above) are
sound:
\begin{itemize}
\item any method that re-assigns a non-final, owning field of an object
  must be annotated with a \<@CreatesMustCallFor> annotation
  whose expression is a reference to that object.
\item when a non-final, owning field $f$ is re-assigned at statement $s$,
  at the program point before $s$, $f$'s must-call obligations must have been satisfied.
\end{itemize}
\noindent
The first rule ensures that \<close()> is called after the last call
to \<reconnect()>, and the second rule ensures that \<reconnect()>
safely closes \<sock> before re-assigning it. Because the Called Methods Checker
treats calls to an \<@CreatesMustCallFor> method like \<reconnect()> as if the call might
cause arbitrary side-effects, after such a call the only method known to have been
definitely called is the \<@CreatesMustCallFor> method: previous called
methods (including \<close()>) do not appear in the \<@CalledMethods> type qualifier.

% TODO: should this section also include text about unconnected sockets, or is what's here sufficient?


\sectionAndLabel{Ignored exception types}{resource-leak-ignored-exceptions}

The Resource Leak Checker checks that an element's must-call obligations
are fulfilled when that element may go out of scope: at the end of its
lexical scope or when control may be transferred to the end of its lexical
scope, such as via a \<break> or \<continue> statement or via throwing an
exception.  As an example of an exception, consider the following method:

\begin{verbatim}
  void foo() {
    Socket s = ...;
    bar();
    s.close();
  }
\end{verbatim}

If \<bar> is declared to throw an exception, the Resource Leak Checker
warns that a \<Socket> may be leaked.  If \<bar> throws an exception, the
only reference to \<s> is lost, which could lead to a resource leak.

The Resource Leak Checker ignores control flow due to some exceptions.

\begin{itemize}
\item
By default the Resource Leak Checker ignores run-time errors that can occur
unpredictably at most points in the program. For example, the JVM can throw
an \<OutOfMemoryError> on any allocation.  Similarly,
\<ClassCircularityError>, \<ClassFormatError>, and \<NoClassDefFoundError>
may occur at any reference to a class.  Such exceptions usually terminate
the program, and in that case unclosed resources do not matter.
Furthermore, any method can throw \<RuntimeException>, and the Checker
Framework pessimistically assumes one can be thrown at every call site.
Accounting for such exceptions would lead to vast numbers of
false positive warnings, so the Resource Leak Checker assumes they are
never thrown.  Strictly speaking, this is an unsoundness:  it can lead to
false negatives (missed resource leaks) if the programmer catches these
exceptions, which is a discouraged practice.

\item
The Resource Leak Checker also ignores exception types that can be verified
to never occur.  In particular, the Resource Leak Checker ignores \<NullPointerException>s
(use the Nullness Checker, \chapterpageref{nullness-checker}) and
\<ArrayIndexOutOfBoundsException>s and \<NegativeArraySizeException>s (use the Index
Checker, \chapterpageref{index-checker}). Other exception types may be added to this
list in the future.  Please let us know if there is a type that you think should
be ignored by filing an issue listing both the exception type and the
verification tool.
\end{itemize}

The set of ignored exception types can be controlled with the option
\<-AresourceLeakIgnoredExceptions=...>.  The option takes a comma-separated list of
fully-qualified exception types.  A type can be prefixed with \<=> to ignore exactly
that type and not its subclasses.  For example, for a very pedantic set of ignored
exceptions use:

\begin{verbatim}
  -AresourceLeakIgnoredExceptions=java.lang.Error, =java.lang.NullPointerException
\end{verbatim}

which ignores \<java.lang.Error> (and all its subclasses) as well as
\<java.lang.NullPointerException> (but not its subclasses).

The keyword \<default> will expand to the default set of ignored exceptions.  So,
to add an additional exception to the set of ignored exceptions, use:

\begin{verbatim}
  -AresourceLeakIgnoredExceptions=default,package.MyCustomException
\end{verbatim}

\sectionAndLabel{Errors about field initialization}{resource-leak-field-initialization}

% Arguably, this is working around a bug in the
% MustCallConsistencyAnalyzer, which could be improved to avoid issuing
% these false positive warnings.

The Resource Leak Checker warns about re-assignments to owning fields,
because the value that was overwritten might not have had its obligations
satisfied.  Such a warning is not necessary on the first assignment to a
field, since the field had no content before the assignment.  Sometimes,
the Resource Leak Checker is unable to determine that an assignment is the
first one, so it conservatively assumes the assignment is a re-assignment
and issues an error.

One way to prevent this false positive warning is to declare the field as \<final>.

Alternately, to suppress all warnings related to field assignments in the
constructor and in initializer blocks, pass the
\<-ApermitInitializationLeak> command-line argument.  This makes the
checker unsound:  the Resource Leak Checker will not warn if the constructor
and initializers set a field more than once.  The amount of leakage is
limited to how many times the field is set.

\sectionAndLabel{Errors about unknown must call obligations}{resource-leak-generic-unknown}

The Resource Leak Checker issues a \<required.method.not.known> error
when a variable with the type \<@MustCallUnknown> has a must call obligation.
\<@MustCallUnknown> rarely occurs, but if you encounter this error usually
the right thing to do is to write an explicit \<@MustCall> annotation
on the indicated expression (e.g., as a cast), because the Must Call Checker
will only use \<@MustCallUnknown> as a default when encountering a language
feature that it is unable to reason about.

\sectionAndLabel{Collections of resources}{resource-leak-collections}

The Resource Leak Checker handles homogeneous collections of resources. In a homogeneous collection, every element
has exactly the same must-call and called-methods properties. Instances of \texttt{java.util.Iterable} are supported;
this section calls those ``collections''. Usage of \texttt{java.util.Iterator}s over \texttt{java.util.Iterable}s are also supported, but they are not considered collections.

What the checker effectively verifies is that any resource collection is properly disposed of. That is, there is some loop that definitely calls the required methods on all elements of the collection.

Towards that end, the Resource Leak Checker tracks every allocated resource collection in two ways:

\begin{itemize}
  \item at most one owning reference for each underlying resource collection is tracked via an ownership type system. This owning reference may arbitrarily mutate the collection. All other references are considered not owning and are restricted - they can't be used to add or remove elements to and from the collection respectively for example.
  \item a complementary dataflow analysis tracks an obligation for each allocated resource collection. The obligation can be passed on to method parameters, fields, or return values for example, just like the ownership. The difference is that while the type system tracks at most one owner per resource collection allowed to mutate it without restriction, the dataflow analysis tracks at least one obligation per resource collection to ensure it is definitely fulfilled.
\end{itemize}

For the purposes of this checker, \textit{resource collections} are precisely defined, as the objects of interest for both the type system and obligation tracking. A Java expression is a \textit{resource collection} if it is:

\begin{enumerate}
  \item A \texttt{java.util.Iterable} or implementation thereof (which includes the entire Java Collections framework), or a \texttt{java.util.Iterator}, and:
  \item Its type parameter upper bound has non-empty \texttt{MustCall} type.
\end{enumerate}

For example, expressions of type \texttt{Socket[]} or \texttt{Iterable<? extends @MustCallUnknown Object>} are both resource collection, but one of type \texttt{Set<String>} is not.

\subsectionAndLabel{Ownership type system for resource collections}{resource-leak-collections-ownership-types}
Of the two tracking mechanisms described above, the ownership type system is more visible from the user perspective. The type hierarchy is the following:

\begin{verbatim}
                 @NotOwningCollection
                          |
                   @OwningCollection
                          |
           @OwningCollectionWithoutObligation
                          |
               @OwningCollectionBottom
\end{verbatim}

\begin{itemize}
  \item \textbf{@OwningCollectionBottom} is the type of non-\textit{resource collections}. It is thus the type the vast majority of expressions have.
  \item \textbf{@OwningCollectionWithoutObligation} is the type of \textit{resource collection} references that definitely own the underlying collection, but also definitely don't have any open calling obligations on their elements. For example, a freshly allocated resource collection is of this type.
  \item \textbf{@OwningCollection} is the type of resource collection references that definitely own the underlying collection, but might have open calling obligations for their elements.
  \item \textbf{@NotOwningCollection} is the type of resource collection references that might not own the underlying resource collection.
\end{itemize}

Only \texttt{@OwningCollection} and \texttt{@NotOwningCollection} are permitted as user-written annotations. The other types are only used internally.

\subsectionAndLabel{Ownership types in action: defaults and ownership transfer}{resource-leak-collections-types-in-action}

A freshly allocated resource collection always defaults to \texttt{@OwningCollectionWithoutObligation}.

\begin{Verbatim}
List<Socket> sockets = new ArrayList<>();
// `sockets` is now @OwningCollectionWithoutObligation
\end{Verbatim}

When you now add elements into this collection, its type unrefines to \texttt{@OwningCollection}.

\begin{Verbatim}
List<Socket> sockets = new ArrayList<>();
// `sockets` is @OwningCollectionWithoutObligation
for (int i = 0; i < n; i++) {
  sockets.add(new Socket(myHost, myPort + i));
  // `sockets` is @OwningCollection
}
// `sockets` is @OwningCollection
\end{Verbatim}

All methods called on a resource collection that potentially move an element with a calling obligation into the collection change the type of the receiving collection to \texttt{@OwningCollection} and create a tracked obligation for the collection in the dataflow analysis.

For method parameters, the default type of resource collection references is \texttt{@NotOwningCollection}.

\begin{Verbatim}
void m(Iterable<Socket> sockets) {
  // `sockets` is @NotOwningCollection
  ...
}
\end{Verbatim}

This means that by default, a resource collection parameter can only do certain 'safe' operations on the collection that do not add, remove, or overwrite existing elements. But it also means that the parameter does not have any calling obligation for the elements of the collection, and it can accept both owning and not owning collection references due to normal subtyping rules. If an \texttt{OwningCollection} reference is passed to an unannotated parameter, the obligation stays at the call-site.

Annotating the parameter with \texttt{@OwningCollection} overrides this behavior. In this case, the ownership is transferred from the call-site argument to the method parameter. Concretely, this means that the call-site obligation is removed and the parameter takes on the responsibility of disposing of the collection. The type of the argument at call-site is changed to \texttt{@NotOwningCollection} to maintain the invariant that there's at most one owning reference for each resource collection.

\begin{Verbatim}
void m(@OwningCollection Iterable<Socket> sockets) {
  // `sockets` is @OwningCollection
  ...
  // this loop disposes of the collection and fulfills the obligation
  for (Socket s : sockets) {
    try {
      s.close();
    } catch (Exception e) {
      System.out.println(e.stackTrace());
    }
  }
  // `sockets` is @OwningCollectionWithoutObligation again
}
\end{Verbatim}

Resource collection return types default to \texttt{@OwningCollection}. Returning a resource collection to such a return type fulfills the obligation for the method. The obligation and ownership are both transferred to the call-site return expression.

\begin{verbatim}
List<Socket> returnSocketList() {
  List<Socket> socketList = new ArrayList<>();
  // `socketList` is @OwningCollectionWithoutObligation
  socketList.add(new Socket(myHost, myPort));
  // `socketList` is @OwningCollection
  return socketList;
  // the return transfers obligation and ownership to the call-site
}
\end{verbatim}

Writing \texttt{@NotOwningCollection} on the return type overrides this default. In this case, ownership and obligation are not transferred to the call-site and the obligation must be fulfilled or passed on prior to the return.

Ownership is also transferred through assignments. If the right-hand side of an assignment is of type \texttt{@OwningCollection} or \texttt{@OwningCollectionWithoutObligation}, the ownership is transferred to the left-hand side expression, and thus the right-hand side is unrefined to \texttt{@NotOwningCollection} to maintain the invariant of having at most one owning reference per resource collection.

\begin{verbatim}
// `r` is @OwningCollection
l = r;
// `l` is now @OwningCollection, `r` is @NotOwningCollection
\end{verbatim}

Just like the ownership, the obligation is also transferred to the left-hand side.

\subsectionAndLabel{Fulfilling collection obligations}{resource-leak-collections-fulfillment}

To fulfill the obligation of a resource collection, the obligation can be passed on, but at some point it has to be fulfilled by calling the required methods on its elements. Recommended are enhanced \texttt{for} loops, but indexed \texttt{for} loops are also supported with some syntactic restrictions.

Here is an example of such a loop:

\begin{verbatim}
// `socketList` is @OwningCollection and has an obligation for calling `close` on its elements
for (Socket s : socketList) {
  try {
    s.close();
  } catch (Exception e) {
    e.printStackTrace();
  }
}
// `socketList` is @OwningCollectionWithoutObligation and has no obligation
\end{verbatim}

The main effect such a loop has is that the obligations corresponding to the methods it calls on all elements are removed. If the loop calls all required methods - which is almost always the case, since classes usually have at most one \texttt{MustCall} method - the type of the collection is additionally refined to \texttt{@OwningCollectionWithoutObligation}.

To determine the methods such a loop calls, the checker does the following things:

\begin{itemize}
  \item It checks that the loop does not terminate early.
  \item It leverages the \texttt{CalledMethods} analysis to determine the methods definitely called on the loop iterator variable. This means that anything that changes the \texttt{CalledMethods} type of the iterator variable is supported, for example passing it as an argument to a method annotated \texttt{@EnsuresCalledMethods(``m'')}, or using try-with constructs. Nullness checks are also supported.
\end{itemize}

If an indexed \texttt{for} loop is used, the checker must additionally check that the loop does in fact iterate over all elements of the collection, and that the loop doesn't write to the loop variable for example. The checks are the following:

\begin{itemize}
  \item Loop iteration bounds:
    \begin{itemize}
      \item The lower bound of iteration is 0, and
      \item The upper bound of iteration is \texttt{c.size()}
      \item The loop's increment expression is a pre- or post-increment of the loop variable.
      \item The collection element is accessed using \texttt{collection.get(i)}, where \texttt{i} is the loop variable.
    \end{itemize}
  \item The loop does not assign to the collection variable.
  \item The loop does not assign to collection elements.
\end{itemize}

Here is an example using an indexed \texttt{for} loop.

\begin{verbatim}
for (int i = 0; i < socketList.size(); i++) {
  try {
    socketList.get(i).close();
  } catch (Exception e) {}
}
\end{verbatim}

\subsectionAndLabel{Resource collection fields}{resource-leak-collections-fields}
By default, resource collection fields are owned by the enclosing class. Static fields are not supported and an error is reported for a declaration of a static resource collection field.

To verify a resource collection field \textit{field}, the checker looks for the following:
\begin{enumerate}
  \item A \texttt{@MustCall} annotation on the enclosing class. In this case, the class is of type \texttt{@MustCall(``close'')} by subclassing Closeable.
  \item Among the methods in the \texttt{@MustCall} type of the enclosing class (usually there's just one), the checker now looks for the post-condition annotation \texttt{@CollectionFieldDestructor(``field'')}.
  \item The annotation \texttt{@CollectionFieldDestructor(``field'')} asserts that at the method exit, \textit{field} has type \texttt{@OwningCollectionWithoutObligation}, which the checker verifies.
  \item Methods that call 'unsafe' methods creating obligations for the resource collection field, such as \texttt{socketList.add(Socket)} in this case, must have a \texttt{@CreatesMustCallFor(``this'')} annotation.
\end{enumerate}

This now shifts the burden to the client of the \texttt{Aggregator} class to call \texttt{close()} on instances before they leave scope.

Consider the following example:

\begin{verbatim}
class Aggregator extends Closeable {
  List<Socket> socketList = new ArrayList<>();

  public Aggregator() {}

  public Aggregator(@OwningCollection List<Socket> l) {
    // `this.socketList` is @OwningCollectionWithoutObligation
    this.socketList = l;
    // `this.socketList` is @OwningCollection
    // the obligation of `l` has been resolved
  }

  @CreatesMustCallFor("this")
  public void add(@Owning Socket s) {
    // this call creates an obligation for the field.
    // thus, the enclosing method must have a @CreatesMustCallFor("this") annotation.
    socketList.add(s);
  }

  @Override
  @CollectionFieldDestructor("socketList")
  public void close() {
    for (Socket s : socketList) {
      try {
        s.close();
      } catch (Exception e) {
        e.printStackTrace();
      }
    }
  }
}
\end{verbatim}

If a resource collection field is final, the checker doesn't have to verify assignments to it. If it is not final, the check verifies assignments in the following way:
\begin{itemize}
  \item Assignments in initializer blocks are not permitted.
  \item Assignments in a declaration initializer are permitted if the right-hand side is at most \texttt{@OwningCollectionWithoutObligation}.
  \item Thanks to the above two restrictions, it can be assumed that an owned resource collection field is \texttt{@OwningCollectionWithoutObligation} at the beginning of a constructor. An assignment to the field is now allowed if the field is \texttt{@OwningCollectionWithoutObligation} just before the assignment. This condition holds in any other (non-constructor) method too, but the field is \texttt{@OwningCollection} at the method entrance, unlike for a constructor. A legal assignment to an owning resource collection field resolves the obligation of the right-hand side.
\end{itemize}

Resource collection fields owned by the enclosing class are special cases for ownership transfer, as the checker doesn't allow transfering the ownership away from such a field. If for example a member method were able to transfer the ownership of the field to an external method parameter, the checker would have to track that the field is no longer the owner and set its type to \texttt{@NotOwningCollection} for future method invocations on the class instance. For the sake of simplicity, the checker doesn't track this and thus doesn't allow ownership transfers for fields. In particular:

\begin{itemize}
  \item The field cannot be passed as an argument to an \texttt{@OwningCollection} parameter.
  \item The field cannot be returned to an \texttt{@OwningCollection} return type.
  \item If the field is on the right-hand side of an assignment, the left-hand side has type \texttt{@NotOwningCollection} after the assignment, so as to keep the at-most-one-owner invariant.
  \item Any access to the field from outside of the class (or a subclass) is not permitted.
\end{itemize}

If the enclosing class is not intended to take responsibility for the resource collection field, annotate it with @NotOwningCollection. It now behaves just like any other resource collection reference. In particular:

\begin{itemize}
  \item Assigning to such a field does not resolve the obligation of the right-hand side.
  \item The enclosing class does not have to fulfill any obligations on the field.
\end{itemize}

\subsectionAndLabel{Iterators over resource collections}{resource-leak-collections-iterators}
Iterators are frequently used to traverse collections. It is always safe to create an iterator for any non-resource collection. Since an iterator takes a snapshot of the collection at the time it is created, it does not matter what happens with the collection after the iterator is created (else, a \<ConcurrentModificationException> is thrown for the next usage of the iterator).
An iterator created for a resource collection has the same collection ownership type as the resource collection.

\begin{verbatim}
// `socketList` is @OwningCollectionWithoutObligation
Iterator<Socket> socketIterator = socketList.iterator();
// `socketIterator` is @OwningCollectionWithoutObligation
\end{verbatim}

The only concering operation an iterator can do is \texttt{remove()}, as the removed element might have open calling obligations.

Just like a \texttt{java.util.Collection} of type \texttt{@NotOwningCollection}, an iterator of this type is not allowed to call \texttt{remove()} at all. Iterators of type \texttt{@OwningCollectionBottom} and \texttt{@OwningCollectionWithoutObligation} can always call \texttt{remove()}. The interesting remaining case is that of an \texttt{@OwningCollection} iterator.
Such an iterator and the values its calls to \texttt{next()} return are tracked with special obligations. This is to ensure that whenever an \texttt{@OwningCollection} iterator calls \texttt{close()}, the value returned by the previous call to \texttt{Iterator.next()} has its \texttt{MustCall} obligations fulfilled.

Here is an example of unsafe iterator usage:

\begin{verbatim}
List<Socket> foo(@OwningCollection List<Socket> list) {
  Iterator<Socket> iter = list.iterator();
  // `iter` is @OwningCollection and must be tracked
  iter.next();
  iter.remove(); // unsafe!
  return list;
}
\end{verbatim}

\begin{verbatim}
 error: [required.method.not.called] @MustCall method close may not have been invoked on iter.next() or any of its aliases.
    iter.next();
             ^
\end{verbatim}

The removed element must have its must-call obligations fulfilled. Here is an example of safe usage:

\begin{verbatim}
List<Socket> foo(@OwningCollection List<Socket> list) {
  Iterator<Socket> iter = list.iterator();
  // `iter` is @OwningCollection and must be tracked
  try {
    iter.next().close();
  } catch (Exception e) {
  }
  iter.remove();
  return list;
}
\end{verbatim}

The returned element may also be stored in a variable first and the fulfillment of the obligation of the returned element may also occur after the call to \texttt{remove()}:

\begin{verbatim}
List<Socket> foo(@OwningCollection List<Socket> list) {
  Iterator<Socket> iter = list.iterator();
  // `iter` is @OwningCollection and must be tracked
  Socket s = iter.next();
  iter.remove();
  try {
    s.close();
  } catch (Exception e) {
  }
  return list;
}
\end{verbatim}

The returned element by \texttt{iter.next()} may have its must-call obligation fulfilled in any way described by the resource leak checker, even by transfering its ownership (for example by storing it in a field or passing it as an \texttt{@OwningCollection} method argument).

Iterators over collections with obligations may also be returned, passed as method or constructor arguments and even be stored in fields.

\subsectionAndLabel{JDK Method Signatures}{resource-leak-collections-jdk-methods}
[[TODO: Add a table with a number of collection/iterable methods and their collection ownership annotations]]


% The Resource Leak Checker cannot verify code that stores a collection of
% resources in a generic collection (e.g., \<java.util.List>) and then
% resolves the obligations of each element of the collection at once (e.g.,
% by iterating over the \<List>).  In the future, the checker will support
% this.  The remainder of this section explains the implementation issues;
% most users can skip it.

% The first implementation issue is that \<@Owning> and \<@NotOwning> are
% declaration annotations rather than type qualifiers, so they cannot be
% written on type arguments. It is possible under the current design to have
% an \code{@Owning List<Socket>}, but not a \code{List<@Owning Socket>}.
% It would be better to make \<@Owning> a type annotation, but this is a
% challenging design problem.

% The second implementation issue is the defaulting rule for \<@MustCall> on
% type variable upper bounds.  Currently, this default is \<@MustCall(\{\})>,
% which prevents many false positives in code with type variables that makes
% no use of resources --- an important design principle.
% However, this defaulting rule does have an unfortunate consequence: it is
% an error to write \code{List<Socket>} or any other type with a concrete
% type argument where the type argument itself isn't \<@MustCall(\{\})>. A programmer who
% needs to write such a type while using the Resource Leak Checker has a few
% choices, all of which have some downsides:

% \begin{itemize}
% \item Write \code{List<? extends Socket>}. This rejects calls to \<add()>
% or other methods that require an instance of the type variable, but it
% preserves some of the behavior (e.g., calls to \<remove()> are permitted).
% This is the best choice most of the time if the \<List> is not intended to
% be owning.
% \item Write \code{List<@MustCall Socket>}. This makes it an error to
% add a Socket to the list, since the type of the Socket is
% \<@MustCall("close")> but the list requires \<@MustCall()>.
% \item Suppress one or more warnings.
% \end{itemize}

% The recommended way to use the Resource Leak Checker in this situation is
% to rewrite the code to avoid a \<List> of owning resources. If rewriting is
% not possible, the programmer will probably need to suppress a warning and
% then verify the code using a method other than the Resource Leak Checker.


\sectionAndLabel{Further reading}{resource-leak-checker-references}

The paper ``Lightweight and Modular Resource Leak
Verification''~\cite{KelloggSSE2021} (ESEC/FSE 2021,
\myurl{https://homes.cs.washington.edu/~mernst/pubs/resource-leak-esecfse2021-abstract.html})
gives more details about the Resource Leak Checker.

The Resource Leak Checker uses a specialized algorithm to infer annotations
when whole program inference (WPI, Section~\ref{whole-program-inference})
is enabled.  The algorithm is described in the paper ``Inference of
Resource Management Specifications''~\cite{ShadabGTEKLLS2023} (OOPSLA 2023,
\myurl{https://homes.cs.washington.edu/~mernst/pubs/resource-inference-oopsla2023-abstract.html}).


% LocalWords:  de subchecker OutputStream MustCall MustCallUnknown RAII Un
% LocalWords:  PolyMustCall InheritableMustCall MultiRandSelector callsite
% LocalWords:  Partitioner CalledMethods AnoLightweightOwnership NotOwning
% LocalWords:  AnoResourceAliasing MustCallAlias AnoCreatesMustCallFor
% LocalWords:  EnsuresCalledMethods CreatesMustCallFor expr Verification''
% LocalWords:  closeSocket destructor'' destructor BufferedOutputStream
% LocalWords:  AnoResourceAliases createsmustcallfor SocketContainer
% LocalWords:  CalledMethodsBottom ArrayIndexOutOfBoundsException
% LocalWords:  NegativeArraySizeException
