\htmlhr
\chapterAndLabel{Resource Leak Checker for must-call obligations}{resource-leak-checker}

The Resource Leak Checker guarantees that the program fulfills every object's
must-call obligations before the object is de-allocated.

A resource leak occurs when a program does not explicitly dispose of some finite
underlying resource, such as a socket, file descriptor, or database connection.  To dispose
of the resource, the program should call some method on an object.
(De-allocating or garbage-collecting the object is not sufficient.)  For
example, the program must call \<close()> on every object that implements the
interface \<java.io.Closeable>.

The Resource Leak Checker can check any property of the form ``the programmer
must call each method in a set of methods \emph{M} at least once
on object \emph{O} before \emph{O} is de-allocated''.  For resource leaks,
by default \emph{M} is the set containing
\<close()> and \emph{O} is any object that implements the interface
\<java.io.Closeable>.  You can extend this guarantee to other types and methods
by writing \<@MustCall> or \<@InheritableMustCall> annotations, as described in
Section~\ref{must-call-annotations}.

The Resource Leak Checker works in three stages:
\begin{enumerate}
\item The Must Call Checker (\chapterpageref{must-call-checker})
  over-approximates each expression's must-call methods as a
  \refqualclass{checker/mustcall/qual}{MustCall} type.
\item The Called Methods Checker (\chapterpageref{called-methods-checker})
  under-approximates each expression's definitely-called methods as a
  \refqualclass{checker/calledmethods/qual}{CalledMethods} type.
\item When any program element goes out of scope (i.e., it is ready to be
  de-allocated), the Resource Leak Checker compares the types
  \<@MustCall(\emph{MC})> and \<@CalledMethods(\emph{CM})>.  It reports an error
  if there exists some method in \emph{MC} that is not in \emph{CM}.
\end{enumerate}

The paper
``Lightweight and Modular Resource Leak Verification''~\cite{KelloggSSE2021} (ESEC/FSE 2021,
\myurl{https://homes.cs.washington.edu/~mernst/pubs/resource-leak-esecfse2021-abstract.html})
gives more details about the Resource Leak Checker.


\sectionAndLabel{How to run the Resource Leak Checker}{resource-leak-run-checker}

Run one of these lines:

\begin{Verbatim}
javac -processor resourceleak MyFile.java ...
javac -processor org.checkerframework.checker.resourceleak.ResourceLeakChecker MyFile.java ...
\end{Verbatim}

The Resource Leak Checker supports all the command-line arguments
listed in Section~\ref{called-methods-run-checker} for
the Called Methods Checker, plus three others:

\begin{description}
\item[\<-ApermitStaticOwning>]
  See Section~\ref{resource-leak-owning-fields}.
\item[\<-AresourceLeakIgnoredExceptions=...>]
  See Section~\ref{resource-leak-ignored-exceptions}.
\item[\<-ApermitInitializationLeak>]
  See Section~\ref{resource-leak-field-initialization}.
%% TODO: Uncomment when the feature is ready to be publicized.
% \item[\<-AenableWpiForRlc>]
%   See Section~\ref{resource-leak-checker-inference-algo}.
\end{description}

If you are running the Resource Leak Checker, then there is no need to run
the Must Call Checker (\chapterpageref{must-call-checker}), because the
Resource Leak Checker does so automatically.


\sectionAndLabel{Resource Leak Checker annotations}{resource-leak-annotations}

The Resource Leak Checker relies on the type qualifiers of two other checkers:
the Must Call Checker (Section~\ref{must-call-annotations}) and
the Called Methods Checker (Section~\ref{called-methods-spec}). You might need
to write qualifiers from either type hierarchy. The most common annotations from
these checkers that you might need to write are:

\begin{description}

\item[\refqualclasswithparams{checker/mustcall/qual}{MustCall}{String[] value}]
for example on an element with compile-time type \<Object> that might contain a \<Socket>.
See Section~\ref{must-call-annotations}.

\item[\refqualclasswithparams{checker/mustcallonelements/qual}{MustCallOnElements}{String[] value}]
expresses the must-call obligations of the elements of an array or collection. Rarely written, most common use case is to specify the obligations of an \<@OwningCollection> method parameter. See Section~\ref{resource-leak-arrays-mcoe-cmoe}.

\item[\refqualclasswithparams{checker/mustcall/qual}{InheritableMustCall}{String[] value}]
on any classes defined in your program that have must-call obligations. See Section~\ref{must-call-on-class}.

\item[\refqualclass{checker/calledmethods/qual}{EnsuresCalledMethods} and/or
      \refqualclass{checker/calledmethods/qual}{EnsuresCalledMethodsOnException}]
on a method in your code that fulfills a must-call obligation of one of its parameters or of a field.
See Section~\ref{called-methods-ensurescalledmethods}.

\item[\refqualclass{checker/calledmethodsonelements/qual}{EnsuresCalledMethodsOnElements} and/or
      \refqualclass{checker/calledmethodsonelements/qual}{EnsuresCalledMethodsOnElementsOnException}]
on a method in your code that fulfills a \<@MustCallOnElements> obligation of one of its array/collection parameters or of an \<@OwningCollection> field. See Section~\ref{resource-leak-arrays-mcoe-cmoe}.

\end{description}

The Resource Leak Checker supports annotations that express
aliasing patterns related to resource leaks:

\begin{description}

\item[\refqualclass{checker/mustcall/qual}{Owning}]
\item[\refqualclass{checker/mustcall/qual}{NotOwning}]
  expresses ownership.  When two aliases exist to the same Java object,
  \<@Owning> and \<@NotOwning> indicate which of the two is responsible for
  fulfilling must-call obligations.
  Constructor results are always \<@Owning>. Method returns default to
  \<@Owning>.  Formal parameters and fields default to \<@NotOwning>.
  For more details, see Section~\ref{resource-leak-ownership}.
\item[\refqualclass{checker/mustcallonelements/qual}{OwningCollection}]
  Every array or collection that is not explicitly annotated as \<@OwningCollection> is not owning. It is a compile-time error if an expression with a must-call obligation is assigned into an element of an array or collection that does not have an \<@OwningCollection> annotation. For more details, see Section~\ref{resource-leak-arrays} and Section~\ref{resource-leak-collections}.

\item[\refqualclass{checker/mustcall/qual}{MustCallAlias}]
  represents a ``resource-aliasing'' relationship.  Resource aliases are
  distinct Java objects that control the same resource(s):
  fulfilling the must-call obligations of one object also
  fulfills the obligations of the other object.  For details,
  see Section~\ref{resource-leak-resource-alias}.

\end{description}

The Resource Leak Checker also supports an annotation to permit re-assigning
fields or re-opening resources:

\begin{description}

\item[\refqualclasswithparams{checker/mustcall/qual}{CreatesMustCallFor}{String value}]
  is a declaration annotation that indicates that after a call to a method
  with this annotation none of the must-call obligations of the in-scope, owning expression
  listed in \<value> have been met.
  In other words, the annotated method ``resets'' the must-call obligations of the expression.
  Multiple \<@CreatesMustCallFor>
  annotations can be written on the same method.  Section~\ref{resource-leak-createsmustcallfor}
  explains how this annotation permits re-assignment of owning
  fields or the re-opening of resources.

\end{description}


\sectionAndLabel{Example of how safe resource usage is verified}{resource-leak-example}

Consider the following example of safe use of a \<Socket>, in which the comments indicate the
inferred Must Call and Called Methods type qualifiers for \<s>:
\begin{verbatim}
{
  Socket s = null;
  // 1. @MustCall({}) @CalledMethodsBottom
  try {
    s = new Socket(myHost, myPort);
    // 2. @MustCall("close") @CalledMethods({})
  } catch (Exception e) {
    // do nothing
  } finally {
    if (s != null) {
      s.close();
      // 3. @MustCall("close") @CalledMethods("close")
    } else {
      // do nothing
      // 4. @MustCall("close") @CalledMethodsBottom
    }
    // 5. @MustCall("close") @CalledMethods("close")
  }
  // 6. @MustCall("close") @CalledMethods("close")
}
\end{verbatim}

At point 1, \<s>'s type qualifiers are the type qualifiers of \<null>:
\<null> has no must-call obligations (\<@MustCall(\{\})>),
and methods cannot be called on it (\<@CalledMethodsBottom>).

At point 2, \<s> is a new \<Socket> object, which
has a must-call obligation (\<@MustCall("close")>)
and has had no methods called on it (\<@CalledMethods(\{\})>).

At point 3, \<close()> has definitely been called on \<s>, so
\<s>'s Called Methods type is updated. Note that the Must Call type
does not change.

At point 4, \<s> is definitely \<null> and its type is adjusted accordingly.

At point 5, \<s>'s type is the least upper bound of the types at points 3
and 4.

At point 6, \<s> goes out of scope.  The Resource Leak Checker reports a
\<required.method.not.called> error if the Must Call set contains any
element that the Called Methods set does not.

\sectionAndLabel{Aliased references and ownership transfer}{resource-leak-ownership}

Resource leak checking is complicated by aliasing.  Multiple expressions
may evaluate to the same Java object, but each object only needs to be
closed once.  (Section~\ref{resource-leak-resource-alias} describes a
related situation called ``resource aliasing'', when multiple Java objects
refer to the same underlying resource.)

For example, consider the following code that safely closes a \<Socket>:

\begin{verbatim}
  void example(String myHost, int myPort) throws IOException {
    Socket s = new Socket(myHost, myPort);
    closeSocket(s);
  }
  void closeSocket(@Owning @MustCall("close") Socket t) {
    try {
      t.close();
    } catch (IOException e) {
      e.printStackTrace();
    }
  }
\end{verbatim}

There are two aliases for a socket object: \<s> in \<example()> and \<t> in
\<closeSocket()>.  Ordinarily, the Resource Leak Checker requires that
\<close()> is called on every expression of type \<Socket>, but that is not
necessary here.  The Resource Leak Checker should not warn when
\<s> goes out of scope in \<example()>, because \<closeSocket()> takes ownership
of the socket --- that is, \<closeSocket()> takes responsibility for closing
it. The \<@Owning> annotation on \<t>'s declaration expresses this fact; it
tells the Resource Leak Checker that \<t> is the reference that must be
closed, and its alias \<s> need not be closed.

Constructor returns are always \<@Owning>.
Method returns default to \<@Owning>,
and parameters and fields default to \<@NotOwning>. This treatment of parameter and
return types ensures sound handling of unannotated third-party libraries: any
object returned from such a library will be tracked by default, and the checker
never assumes that passing an object to an unannotated library will satisfy its obligations.

\<@Owning> and \<@NotOwning> always \emph{transfer} must-call obligations: must-call
obligations are conserved (i.e., neither created nor destroyed) by ownership annotations.
Writing \<@Owning> or \<@NotOwning> can never make the checker
unsound:  a real warning can never be hidden by them.
As with any annotation, incorrect or missing annotations can lead to false positive warnings.

\subsectionAndLabel{Owning parameters and exceptions}{resource-leak-owning-parameters-and-exceptions}

When \<@Owning> is written on a method parameter, the method only takes ownership of the
parameter when it returns normally.  In this example, the Resource Leak Checker will report
an error in the \<example> method and allow the definition of \<closeSocket>:

\begin{verbatim}
  void example(String myHost, int myPort) throws Exception {
    // Error: `s` is not closed on all paths
    Socket s = new Socket(myHost, myPort);

    // `closeSocket` does not have to close `s` when it throws IOException.
    // Instead, this method has to catch the exception and close `s`.
    closeSocket(s);
  }

  void closeSocket(@Owning Socket t) throws IOException {
    throw new IOException();
  }
\end{verbatim}

Sometimes a method really does promise to call some methods on an \<@Owning> parameter,
even if it throws an exception.  The annotation \<@EnsuresCalledMethodsOnException> can
overcome this limitation.  For example, a constructor that throws an exception might
choose to close an \<@Owning> parameter instead of letting ownership remain with the caller:

\begin{verbatim}
  @EnsuresCalledMethodsOnException(value = "#1", methods = "close")
  public Constructor(@Owning Closeable resource) {
    this.resource = resource;
    try {
      initialize();
    } catch (Exception e) {
      resource.close();
      throw e;
    }
  }
\end{verbatim}

\subsectionAndLabel{Owning fields}{resource-leak-owning-fields}

Unannotated fields are treated as non-owning.

For \textbf{final, non-static owning fields},
the Resource Leak Checker enforces the ``resource acquisition is
initialization (RAII)'' programming idiom.  Some
destructor-like method \<d()> must satisfy the field's must-call obligation
(and this fact must be expressed via a \<@EnsuresCalledMethods> annotation on \<d()>),
and the enclosing class must have a \<@MustCall("d")> obligation to
ensure the destructor is called. In addition to the \<@EnsuresCalledMethods> annotation,
which guarantees that the field(s) it references have their must-call obligations satisfied
on non-exceptional paths, the Resource Leak Checker requires those fields to have their must-call
obligations satisfied on all paths in (only) the destructor, and will issue a \<destructor.exceptional.postcondition>
error if they are not satisfied. Resolve this error by ensuring that the required methods are called
on all exceptional paths.

\textbf{Non-final, non-static owning fields} usually require one or more \<@CreatesMustCallFor> annotations
when they might be re-assigned. See Section~\ref{resource-leak-createsmustcallfor} for
more details on how to annotate a non-final, non-static owning field.

Owning fields are treated slightly differently in constructors versus normal methods.
In normal methods, assigning a value to an owning field always satisfies the object's
must-call obligations.  However, within a constructor, those obligations are only
satisfied if the constructor returns normally.  If the constructor throws an exception,
the constructed object will not be accessible afterward, and therefore its fields need
to be closed before it exits.

This constructor safely closes the object it allocates before throwing an exception:
\begin{verbatim}
  private final @Owning Socket socket;

  public ConstructorThatCanThrow() throws IOException {
    Socket s = new Socket(myHost, myPort);
    try {
      initialize(s); // may throw IOException
    } catch (Exception e) {
      s.close();
      throw e;
    }
    this.socket = s;
  }
\end{verbatim}

An assignment to a \textbf{static owning field} does not satisfy a
must-call obligation; for example,

\begin{smaller}
\begin{Verbatim}
  static @Owning PrintWriter debugLog = new PrintWriter("debug.log");
\end{Verbatim}
\end{smaller}

\noindent
The Resource Leak Checker issues a warning about every assignment of an
object with a must-call annotation into a static owning field,
indicating that the obligation of the field's content might not be
satisfied.  When those fields are used throughout execution, until the
program exits, there is no good place to dispose of them, so these warnings
might not be useful.  The \<-ApermitStaticOwning> command-line argument
suppresses warnings related to static owning fields.  This can help in
checking legacy code.  It permits only a small number of resource retained
throughout execution, related to the number of such fields and assignments
to them.


\sectionAndLabel{Resource aliasing}{resource-leak-resource-alias}

A \emph{resource alias} set is a set of Java objects that
correspond to the same underlying system resource.
Calling a must-call method on any member of a resource-alias set
fulfills that obligation for all members of the set.
Members of the set may have different Java types.

Programmers most often encounter resource aliasing when using \emph{wrapper types}.
For example, the Java \<Buffered\-Output\-Stream> wrapper adds buffering to a
delegate stream.
The wrapper's \<close()> method invokes \<close()> on the delegate.  Calling
\<close()> on either object has the same effect:  it closes the underlying resource.

A resource aliasing relationship is expressed in source code via a pair of \<@MustCallAlias> annotations:
one on a parameter of a method or constructor, and another on its return type.
For example, the annotated JDK contains this constructor of \<BufferedOutputStream>:
\begin{Verbatim}
@MustCallAlias BufferedOutputStream(@MustCallAlias OutputStream out);
\end{Verbatim}

When a pair of \<@MustCallAlias> annotations is written on a method or constructor \<m>'s return type
and its parameter \<p>, the Resource Leak Checker requires one of the following:
\begin{enumerate}
\item \<p> is passed to another method or constructor (including \<super>) in a
  \<@MustCallAlias> position, and \<m> returns that method's result, or
\item \<p> is stored in the only \<@Owning> field of the enclosing class (a class with more than one
  \<@Owning> field cannot have a resource alias relationship).
\end{enumerate}

\subsectionAndLabel{A complete wrapper type example}{resource-leak-wrapper-type-example}

Here is a complete example of a type \<InputStreamWrapper> that wraps an \<InputStream> as a resource alias.  Defining a wrapper type typically involves combined usage of \<@InheritableMustCall>, \<@EnsuresCalledMethods>, an \<@Owning> field, and \<@MustCallAlias>.  The \<test> method shows that the checker is able to verify code that releases an \<InputStream> using either the \<InputStream> directly or a wrapping \<InputStreamWrapper>.

\begin{verbatim}
@InheritableMustCall("dispose")
class InputStreamWrapper {
  private final @Owning InputStream stream;

  @MustCallAlias InputStreamWrapper(@MustCallAlias InputStream stream) {
    this.stream = stream;
  }

  @EnsuresCalledMethods(value = "this.stream", methods = "close")
  public void dispose() throws IOException {
    this.stream.close();
  }

  /** Shows that either the stream or the wrapper can be closed. */
  static void test(@Owning InputStream stream, boolean b) throws IOException {
    InputStreamWrapper wrapper = new InputStreamWrapper(stream);
    if (b) {
      stream.close();
    } else {
      wrapper.dispose();
    }
  }
}
\end{verbatim}


\sectionAndLabel{Creating obligations (how to re-assign a non-final owning field)}{resource-leak-createsmustcallfor}

Consider a class that has must-call obligations; that is, the class
declaration is annotated with \<@MustCall(...)>.
Every constructor implicitly creates obligations for the newly-created object.
Non-constructor methods may also create obligations
when re-assigning non-final owning fields or allocating
new system-level resources.

A post-condition annotation,
\<@CreatesMustCallFor>,
indicates for which expression an obligation is created.
If you write \<@CreatesMustCallFor(>\emph{T}\<)> on a method \emph{N} that
overrides a method \emph{M}, then \emph{M} must also be annotated as
\<@CreatesMustCallFor(>\emph{T}\<)>.  (\emph{M} may also have other
\<@CreatesMustCallFor> annotations that \emph{N} does not.)

\<@CreatesMustCallFor> allows the Resource Leak Checker to verify uses of non-final fields
that contain a resource, even if they are re-assigned. Consider
the following example:

\begin{verbatim}
  @MustCall("close") // default qualifier for uses of SocketContainer
  class SocketContainer {
    private @Owning Socket sock;

    public SocketContainer() { sock = ...; }

    void close() { sock.close() };

    @CreatesMustCallFor("this")
    void reconnect() {
      if (!sock.isClosed()) {
        sock.close();
      }
      sock = ...;
    }
  }
\end{verbatim}

In the lifetime of a \<SocketContainer> object, \<sock>
might be re-assigned arbitrarily many times: once at each
call to \<reconnect()>. This code is safe, however: \<reconnect()>
ensures that \<sock> is closed before re-assigning it.

Sections~\ref{resource-leak-createsmustcallfor-callsite}
and~\ref{resource-leak-createsmustcallfor-declaration}
explain how the Resource Leak Checker verifies uses and declarations of
methods annotated with \<@CreatesMustCallFor>.


\subsectionAndLabel{Requirements at a call site of a \<@CreatesMustCallFor> method}{resource-leak-createsmustcallfor-callsite}

At a call site to a method annotated as
\<@CreatesMustCallFor(>\emph{expr}\<)>, the Resource Leak Checker:
\begin{enumerate}
\item
  Treats any existing \<@MustCall> obligations of \emph{expr} as \emph{satisfied},
\item
  Creates a fresh obligation to check, as if \emph{expr} was assigned to a newly-allocated
  object (i.e. as if \emph{expr} were a constructor result).
\item
  Un-refines the type in the Called Methods Checker's type hierarchy for \emph{expr} to
  \<@CalledMethods(\{\})>, if it had any other Called Methods type.
\item
  Requires that the expression corresponding to \emph{expr} (that is, \emph{expr}
  viewpoint-adapted to the method call site) is owned; that is, it is
  annotated or defaulted as \<@Owning>.  Otherwise, the checker
  will issue a \<reset.not.owning> error at the call-site. You can avoid this
  error by extracting \emph{expr} into a new local variable (because
  locals are \<@Owning> by default) and replacing all instances of \emph{expr}
  in the call with references to the new local variable.
\end{enumerate}

Treating the obligation before the call as satisfied is sound: the
checker creates a new obligation for calls to \<@CreatesMustCallFor> methods,
and the Must Call Checker (\chapterpageref{must-call-checker}) ensures the
\<@MustCall> type for the target expression will have a \emph{superset} of any methods
present before the call. Intuitively, calling an \<@CreatesMustCallFor> method
``resets'' the obligations of the target expression, so whether they were satisfied before
the call or not is irrelevant.

If an \<@CreatesMustCallFor>
method \emph{n} is invoked within a method \emph{m} that has an \<@CreatesMustCallFor> annotation,
and the \<@CreatesMustCallFor> annotations on \emph{n} and \emph{m} have
the same target expression---imposing the obligation produced by calling \emph{n} on the caller of \emph{m}---then
the newly-created obligation is treated as satisfied immediately
at the call-site of \emph{n} in the body of \emph{m} (because it is imposed at call-sites of \emph{m}
instead).


\subsectionAndLabel{Requirements at a declaration of a \<@CreatesMustCallFor> method}{resource-leak-createsmustcallfor-declaration}

Any method that re-assigns a non-final, owning field of some object \<obj>
must be annotated \<@CreatesMustCallFor("obj")>.
Other methods may also be annotated with \<@CreatesMustCallFor>.

The Resource Leak Checker enforces two rules to ensure that re-assignments
to non-final, owning fields (like \<sock> in method \<reconnect> above) are
sound:
\begin{itemize}
\item any method that re-assigns a non-final, owning field of an object
  must be annotated with a \<@CreatesMustCallFor> annotation
  whose expression is a reference to that object.
\item when a non-final, owning field $f$ is re-assigned at statement $s$,
  at the program point before $s$, $f$'s must-call obligations must have been satisfied.
\end{itemize}
\noindent
The first rule ensures that \<close()> is called after the last call
to \<reconnect()>, and the second rule ensures that \<reconnect()>
safely closes \<sock> before re-assigning it. Because the Called Methods Checker
treats calls to an \<@CreatesMustCallFor> method like \<reconnect()> as if the call might
cause arbitrary side-effects, after such a call the only method known to have been
definitely called is the \<@CreatesMustCallFor> method: previous called
methods (including \<close()>) do not appear in the \<@CalledMethods> type qualifier.

% TODO: should this section also include text about unconnected sockets, or is what's here sufficient?


\sectionAndLabel{Ignored exception types}{resource-leak-ignored-exceptions}

The Resource Leak Checker checks that an element's must-call obligations
are fulfilled when that element may go out of scope: at the end of its
lexical scope or when control may be transferred to the end of its lexical
scope, such as via a \<break> or \<continue> statement or via throwing an
exception.  As an example of an exception, consider the following method:

\begin{verbatim}
  void foo() {
    Socket s = ...;
    bar();
    s.close();
  }
\end{verbatim}

If \<bar> is declared to throw an exception, the Resource Leak Checker
warns that a \<Socket> may be leaked.  If \<bar> throws an exception, the
only reference to \<s> is lost, which could lead to a resource leak.

The Resource Leak Checker ignores control flow due to some exceptions.

\begin{itemize}
\item
By default the Resource Leak Checker ignores run-time errors that can occur
unpredictably at most points in the program. For example, the JVM can throw
an \<OutOfMemoryError> on any allocation.  Similarly,
\<ClassCircularityError>, \<ClassFormatError>, and \<NoClassDefFoundError>
may occur at any reference to a class.  Such exceptions usually terminate
the program, and in that case unclosed resources do not matter.
Furthermore, any method can throw \<RuntimeException>, and the Checker
Framework pessimistically assumes one can be thrown at every call site.
Accounting for such exceptions would lead to vast numbers of
false positive warnings, so the Resource Leak Checker assumes they are
never thrown.  Strictly speaking, this is an unsoundness:  it can lead to
false negatives (missed resource leaks) if the programmer catches these
exceptions, which is a discouraged practice.

\item
The Resource Leak Checker also ignores exception types that can be verified
to never occur.  In particular, the Resource Leak Checker ignores \<NullPointerException>s
(use the Nullness Checker, \chapterpageref{nullness-checker}) and
\<ArrayIndexOutOfBoundsException>s and \<NegativeArraySizeException>s (use the Index
Checker, \chapterpageref{index-checker}). Other exception types may be added to this
list in the future.  Please let us know if there is a type that you think should
be ignored by filing an issue listing both the exception type and the
verification tool.
\end{itemize}

The set of ignored exception types can be controlled with the option
\<-AresourceLeakIgnoredExceptions=...>.  The option takes a comma-separated list of
fully-qualified exception types.  A type can be prefixed with \<=> to ignore exactly
that type and not its subclasses.  For example, for a very pedantic set of ignored
exceptions use:

\begin{verbatim}
  -AresourceLeakIgnoredExceptions=java.lang.Error, =java.lang.NullPointerException
\end{verbatim}

which ignores \<java.lang.Error> (and all its subclasses) as well as
\<java.lang.NullPointerException> (but not its subclasses).

The keyword \<default> will expand to the default set of ignored exceptions.  So,
to add an additional exception to the set of ignored exceptions, use:

\begin{verbatim}
  -AresourceLeakIgnoredExceptions=default,package.MyCustomException
\end{verbatim}

\sectionAndLabel{Errors about field initialization}{resource-leak-field-initialization}

% Arguably, this is working around a bug in the
% MustCallConsistencyAnalyzer, which could be improved to avoid issuing
% these false positive warnings.

The Resource Leak Checker warns about re-assignments to owning fields,
because the value that was overwritten might not have had its obligations
satisfied.  Such a warning is not necessary on the first assignment to a
field, since the field had no content before the assignment.  Sometimes,
the Resource Leak Checker is unable to determine that an assignment is the
first one, so it conservatively assumes the assignment is a re-assignment
and issues an error.

One way to prevent this false positive warning is to declare the field as \<final>.

Alternately, to suppress all warnings related to field assignments in the
constructor and in initializer blocks, pass the
\<-ApermitInitializationLeak> command-line argument.  This makes the
checker unsound:  the Resource Leak Checker will not warn if the constructor
and initializers set a field more than once.  The amount of leakage is
limited to how many times the field is set.

\sectionAndLabel{Errors about unknown must call obligations}{resource-leak-generic-unknown}

The Resource Leak Checker issues a \<required.method.not.known> error
when a variable with the type \<@MustCallUnknown> has a must call obligation.
\<@MustCallUnknown> rarely occurs, but if you encounter this error usually
the right thing to do is to write an explicit \<@MustCall> annotation
on the indicated expression (e.g., as a cast), because the Must Call Checker
will only use \<@MustCallUnknown> as a default when encountering a language
feature that it is unable to reason about.


\sectionAndLabel{Collections of resources}{resource-leak-collections}

The Resource Leak Checker handles homogeneous collections of resources,
such as arrays, \<List>s, etc.  In a homogeneous collection, every element
has exactly the same must-call and called-methods properties.

The only way to operate on a homogeneous collection while maintaining
homogeneneity is by using a loop.  The Resource Leak Checker permits
specific types of loops, as explained below.


\subsectionAndLabel{\<@MustCall> and \<@CalledMethods> for collections}{resource-leak-collections-mcoe-cmoe}

\subsectionAndLabel{Declaration and assignment of resource-holding collections}{resource-leak-collections-declaration}
Annotating the collection with the \<@OwningCollection> declaration annotation allows it to take on such obligations by using a pattern-matched assignment syntax:

\begin{verbatim}
@OwningCollection Socket[] s = new Socket[n];
// 1. @MustCallOnElements({}) @CalledMethodsOnElements({})
for (int i = 0; i < s.length; i++) {
  try {
    s[i] = new Socket(myHost, myPort);
  } catch (Exception e) {
    // do nothing
  }
}
// 2. @MustCallOnElements({"close"}) @CalledMethodsOnElements({})
\end{verbatim}
For the checker to accept the assignment loop, there are some syntactic rules to follow:
\begin{itemize}
  \item The loop must iterate from 0 to \<s.length> or \<n>, where \<n> is the effectively final variable used for the initialization of the collection size.
  \item The update must be either a pre-increment or post-increment.
  \item There must be exactly one statement in the body (which may be a try-catch construct) that contains the assignment.
  \item The right-hand side of the assignment must be the immediate construction of the object (not simply any expression evaluating to one).
  \item If there is a try-catch construct, the catch and finally blocks may not throw any exceptions, break out of the loop, call any methods or have any writes or return statements.
  \item In order to assign the elements of an \<@OwningCollection>, the collection must have no open must-call obligations on its elements before the loop, meaning that its \<@MustCallOnElements> values must be a subset of the \<@CalledMethodsOnElements> values.
\end{itemize}

\noindent Any other assignment to the elements of an \<@OwningCollection>, outside of such a loop, is prohibited.

\noindent An \<@OwningCollection> reference may be assigned an arbitrary amount of times, but only to allocate a new collection (and only if the collection has no open \<@MustCallOnElements> obligations). Any other assignment with the left-hand side being an \<@OwningCollection> reference is prohibited. In particular, an aliasing of two \<@OwningCollection> references is prohibited:

\begin{verbatim}
@OwningCollection Socket[] s1 = new Socket[n];
@OwningCollection Socket[] s2 = new Socket[n];
s2 = s1; // prohibited assignment
\end{verbatim}

In general, aliasing of any kind for \<@OwningCollection> variables is not permitted. In particular, a non-\<@OwningCollection> variable may also not be assigned to an \<@OwningCollection> variable. This implies that an \<@OwningCollection> variable is always the only reference to the underlying collection within a procedure. This is not the case for method calls, when an \<@OwningCollection> reference is passed as an argument to a method call, which is discussed in detail in Section~\ref{resource-leak-collections-parameters}. In short, if the \<@OwningCollection> is passed to a constructor, ownership of the collection at call-site is lost and the reference becomes read-only.
\noindent If the method is not a constructor, it promises to fulfill all obligations before the method ends (while a constructor may store the \<@OwningCollection> parameter in a field), which means that at the call-site, the \<@OwningCollection> keeps ownership and does not become a read-only reference.

\subsectionAndLabel{\<@MustCall> and \<@CalledMethods> for collections}{resource-leak-collections-mcoe-cmoe}
\<@MustCallOnElements> and \<@CalledMethodsOnElements> provide similar semantics for collections as \<@MustCall> and \<@CalledMethods> do for objects.

The type hierarchy is the following:

\begin{verbatim}
              MustCallOnElementsUnknown
                           |
            MustCallOnElements({"a", "b"})
               /                     \
MustCallOnElements({"b"})       MustCallOnElements({"a"})
               \                    /
               MustCallOnElements({})
\end{verbatim}
The default type is \<@MustCallOnElements(\{\})> (bottom) for everything except \<@OwningCollection> fields and \<@OwningCollection> parameters. For those, the default \<@MustCallOnElements> value is the \<@MustCall> value of the component type. For example, the default type of an \<@OwningCollection> Socket[] is \<@MustCallOnElements({``close''})>, since Socket has \<@MustCall(``close'')> type. This reduces the need for many manual annotations.

The fulfilled obligations are stored in the \<@CalledMethodsOnElements> type annotation with the following type hierarchy:
\begin{verbatim}
                      CalledMethodsOnElements({})
                      /                        \
CalledMethodsOnElements({"b"})           CalledMethodsOnElements({"a"})
                      \                        /
                 CalledMethodsOnElements({"a", "b"})
                                   |
                    CalledMethodsOnElementsBottom
\end{verbatim}
The default type is \<@CalledMethodsOnElements(\{\})> (top) for any declaration.


\subsectionAndLabel{Declaration and assignment of resource-holding collections}{resource-leak-collections-declaration}

To indicate that each resource in a collection is owned by the collection, write the \<@OwningCollection> declaration annotation:

\begin{Verbatim}
@OwningCollection Socket[] s = new Socket[n];
// The type of `s` is now  @MustCallOnElements({}) @CalledMethodsOnElements({})
\end{Verbatim}

A reference to an owning collection may be re-assigned any number of times.
The right-hand side must create a new collection that has no
\<@MustCallOnElements> obligations.

\begin{verbatim}
@OwningCollection Socket[] s1 = new Socket[n]; // OK
@OwningCollection Socket[] s2 = new Socket[n]; // OK
s2 = new Socket[n]; // OK
s2 = s1; // prohibited, because the right-hand side does not create a new collection
\end{verbatim}

In general, aliasing of \<@OwningCollection> variables is not permitted.
This implies that an \<@OwningCollection> variable is always the only
reference to the underlying collection.  This property can be violated when
an \<@OwningCollection> reference is passed as an argument to a method
call.  Section~\ref{resource-leak-collections-parameters} discusses method
calls.


\subsectionAndLabel{Using loops to operate on resource-holding collections}{resource-leak-collections-loops}

In order to maintain homogeneity, code must use a loop to operate on a
collection of resources.  Any operation (assignment, method calls, etc.)\
on the elements of an \<@OwningCollection>, outside of a loop, is
prohibited.  Furthermore, the loop must not terminate early --- it must
operate upon every element of the collection.

More concretely, the basic loop constraints are:
\begin{itemize}
\item Loop iteration bounds:
  \begin{itemize}
  \item
    The loop is an enhanced \<for> loop (``foreach loop'') over the collection, or
  \item
    The loop is an indexed \<for> loop, such that
    \begin{itemize}
    \item The lower bound of iteration is 0, and
    \item The upper bound of iteration is \<c.length> or \<c.size()> or \<n>, where \<n> is the effectively final variable used for the initialization of the collection size.
    \item The loop contains no assignments to the loop variable.
    \end{itemize}
  \end{itemize}
\item The loop does not terminate early: The loop body contains no \<return> or \<break> statements.
\item The loop does not assign to the collection.
\item The loop does not assign to collection elements, unless it is an
  assignment loop (Section~\ref{resource-leak-collections-assign-elements}).
\end{itemize}

\noindent There are additional constraints for specific types of loops.



\subsectionAndLabel{Assigning elements in a resource-holding collection}{resource-leak-collections-assign-elements}

Here is an example of how to assign the elements of a collection of resources:

\begin{Verbatim}
// Initially, the type of `s` is  @MustCallOnElements({}) @CalledMethodsOnElements({})
for (int i = 0; i < s.length; i++) {
  try {
    s[i] = new Socket(myHost, myPort);
  } catch (Exception e) {
    // do nothing
  }
}
// Now, the type of `s` is @MustCallOnElements({"close"}) @CalledMethodsOnElements({})
\end{Verbatim}

The checker accepts an assignment loop if the following requirements are
satisfied (in addition to the basic loop constraints of Section~\ref{resource-leak-collections-loops}):

\begin{itemize}
  \item The collection elements have no open must-call obligations.  That
  is, the collection's \<@MustCallOnElements> values are a subset of its
  \<@CalledMethodsOnElements> values.
  \item The loop is an indexed \<for> loop (not an enhanced \<for> loop).
  \item There is exactly one statement in the body (which may be a try-catch construct) that contains the assignment.
  \item The right-hand side of the assignment creates a new resource.
    The right-hand side cannot be an expression that might evaluate to an existing resource.
  \item If there is a try-catch construct, the catch and finally blocks may
  not throw any exceptions, call any methods, or have any writes (in
  addition to the basic loop constraints of Section~\ref{resource-leak-collections-loops}).
\end{itemize}


\subsectionAndLabel{Fulfilling \<@MustCallOnElements> obligations}{resource-leak-collection-fulfilling-obligations}

A \<@MustCallOnElements> obligation can be fulfilled via operations on loop
elements such as \<c[i]> or \<c.get(i)>.
The loop has no constraints beyond the basic loop constraints of
Section~\ref{resource-leak-collections-loops}), and its body may call an
arbitrary number of methods on the collection elements.

Here are two examples:

\begin{verbatim}
// Suppose the type of `arr` is initially @MustCallOnElements({"close"}) @CalledMethodsOnElements({})
for (int i = 0; i < arr.length; i++) {
    try {
      arr[i].close();
      arr[i].foo();
    } catch (Exception e) {}
}
// Then the type of `arr` is now. @MustCallOnElements({}) @CalledMethodsOnElements({"close", "foo"})
\end{verbatim}

The obligation can also be fulfilled by calling a method that has an \<@EnsuresCalledMethods> annotation, with the collection element as an argument:
\begin{verbatim}
// Suppose the type of `arr` is initially @MustCallOnElements({"close"}) @CalledMethodsOnElements({})
for (int i = 0; i < arr.length; i++) {
    method(arr[i]); // method calls 'close' on its argument, see below
}
// Then the type of `arr` is now @MustCallOnElements({}) @CalledMethodsOnElements({"close"})

...

@EnsuresCalledMethods(value="#1", methods={"close"})
void method(Socket s) {
    try {
        s.close();
    } catch (Exception e) {}
}
\end{verbatim}

Here is an example written as an enhanced \<for> loop:

\begin{verbatim}
// 1. @MustCallOnElements({"close"}) @CalledMethodsOnElements({})
for (Socket s: arr) {
    try {
        s.close();
    } catch (Exception e) {}
}
// 2. @MustCallOnElements({}) @CalledMethodsOnElements({"close"})
\end{verbatim}


\subsectionAndLabel{\<@OwningCollection> method parameters}{resource-leak-collections-parameters}

When \<@OwningCollection> is written on a method parameter, the default type of the parameter is \<@MustCallOnElements(\textit{obligations})>, where \textit{obligations} is the set of \<@MustCall> obligations of the component type. The user is allowed to put any manual \<@MustCallOnElements> annotation to override this default however.

\begin{verbatim}
void m1(@OwningCollection Socket[] s) {
  // 1. @MustCallOnElements({"close"}) @CalledMethodsOnElements({})
  for (int i = 0; i < s.length; i++) {
    try {
      s.close();
    } catch {}
  }
  // 2. @MustCallOnElements({}) @CalledMethodsOnElements({"close"})
}
\end{verbatim}

To call a method with an \<@OwningCollection> parameter, the argument must also be \<@OwningCollection>. The called method only borrows the ownership of the \<@OwningCollection> parameter. This is because the \<@OwningCollection> parameter cannot leave the method scope and its obligations have to be fulfilled by the time of method exit. Thus, not only can the caller keep the ownership for the passed argument after the method call returned, but due to the calling guarantees, the call also resets the \<@MustCallOnElements> type to an empty value and adds the methods guaranteed to have been called to the \<CalledMethodsOnElements> type:

\begin{verbatim}
// 1. @MustCallOnElements({"close"}) @CalledMethodsOnElements({})
m1(arr);
// 2. @MustCallOnElements({}) @CalledMethodsOnElements({"close"})
\end{verbatim}

After calling \texttt{m1}, the elements of \textit{arr} may be reassigned.

If the parameter is non-\<@OwningCollection>, the argument must also be non-\<@OwningCollection>.

\subsectionAndLabel{\<@OwningCollection> fields}{resource-leak-collections-fields}
\<@OwningCollection> fields must be final, private and non-static. Thus, they can only be assigned either in the constructor or at the declaration site.
The default type of such an collection after declaration is \<@MustCallOnElements(obligation)>, where \textit{obligation} is the set of \<@MustCall> methods of the component type. The default \<@CalledMethodsOnElements> type is the empty \<@CalledMethodsOnElements({})>.
Since the lifetime of a field coincides with the lifetime of the class instance, the obligations of the field must be fulfilled at the time the object itself is deallocated.
Thus, a class with an \<@OwningCollection> field has to have a \<@MustCall("d")> annotation, where \texttt{d()} is some destructor method that fulfills all obligations of the \<@OwningCollection> fields, indicated by an appropriate \<@EnsuresCalledMethodsOnElements> annotation:

\begin{verbatim}
@MustCall("destruct")
class A {
    final @OwningCollection Resource[] arr = new Resource[10];
    ...
    @EnsuresCalledMethodsOnElements(value="arr", methods={"close"})
    public void destruct() {
        for (int i = 0; i < arr.length; i++) {
            arr[i].close();
        }
    }
}
\end{verbatim}

This ensures that any client that constructs an instance of A will have to call \texttt{destruct()} on it, which in turn ensures that the obligations of the field are fulfilled.

The elements of an \<@OwningCollection> field may only be assigned in the constructor and only once. Thus, an \<@OwningCollection> field provides immutability beyond just the final keyword, as even its elements are immutable once the enclosing class is constructed. The collection stored in the field does not necessarily have to be allocated in the constructor, but may also be passed as an argument to the constructor, see Section~\ref{resource-leak-collections-constructor}.

\subsectionAndLabel{\<@OwningCollection> constructor parameters}{resource-leak-collections-constructor}
Constructors are special in the context of this checker, since they're the only methods where \<@OwningCollection> fields may be assigned.

When \<@OwningCollection> is written on a constructor parameter, the default of the parameter is \<@MustCallOnElements(obligations)>, where \textit{obligations} is the set of \<@MustCall> obligations of the component type. It is allowed to put any manual \<@MustCallOnElements> annotation to override this default however.

An \<@OwningCollection> constructor argument may be assigned to an \<@OwningCollection> field, which removes the obligation of the parameter.

\begin{verbatim}
@MustCall({"destruct"})
class A {
  final @OwningCollection Socket[] s;

  public A(@OwningCollection Socket[] arr) {
    // the obligation of arr is closed by assigning into @OwningCollection field s
    s = arr;
  }

  @EnsuresCalledMethodsOnElements(value="s", methods={"destruct"})
  public void destruct() {
    for (int i = 0; i < s.length; i++) {
      try {
        s[i].close();
      } catch {}
    }
  }

}
\end{verbatim}

At the call-site, passing an \<@OwningCollection> argument as an \<@OwningCollection> parameter to a constructor, removes the obligation for the argument. However, its ownership is revoked, which makes it a read-only reference. Writes to the collection elements are forbidden. This read-only status is made visible by setting its \<@MustCallOnElements> type to \<@MustCallOnElementsUnknown>. The elements of the collection may not be reassigned now and it may not be passed to an \<@OwningCollection> parameter. The reference may be reassigned to a new collection however, for which it will have write access again.

\begin{verbatim}
@OwningCollection Socket[] s;
...
// 1. s: @MustCallOnElements({"close"}) @CalledMethodsOnElements({})
A a = new A(s);
// 2. s: @MustCallOnElementsUnknown @CalledMethodsOnElements({})
\end{verbatim}

Now, the elements of \textit{s} may not be reassigned. The following throws an error:

\begin{verbatim}
// s: @MustCallOnElementsUnknown @CalledMethodsOnElements({})
for (int i = 0; i < s.length; i++) {
  try {
    s[i] = new Socket(myHost, myPort);  // this assignment throws an error
  } catch {}
}
\end{verbatim}

To regain ownership, \texttt{s} can be reassigned to a new collection:

\begin{verbatim}
// s: @MustCallOnElementsUnknown @CalledMethodsOnElements({})
s = new Socket[n];
// s: @MustCallOnElements({}) @CalledMethodsOnElements({})
\end{verbatim}

Now, the elements of \textit{s} may be assigned in a loop.

% It is also allowed to have:
% \begin{verbatim}
% for (Resource r : arr) {
%     someMethod(arr);
% }
% \end{verbatim}
% If someMethod is annotated with \<@EnsuresCalledMethodsOnElements(value="#1", methods="x")>, this loop is considered to fulfill the obligation of calling "x" on elements of "arr".

\sectionAndLabel{Collections of resources}{resource-leak-collections}
Collections of resources are handled in a similar way to collections. Specifically, all objects that are assignable to a \<Collection> instance variable may be annotated \<@OwningCollection>, just like an collection intended to hold resources. Since collections and collections that hold resources share the same \<@OwningCollection> annotation, much of their logic is the same. In particular, see Section~\ref{resource-leak-collections-parameters}, Section~\ref{resource-leak-collections-fields} and Section~\ref{resource-leak-collections-constructor} to see how \<@OwningCollection> parameters, fields and constructor parameters are handled. In the following, only the aspects that differ for \<@OwningCollection> collections, as opposed to collections, are detailed.

The type system is the same for collections and collections (see Section~\ref{resource-leak-collections-mcoe-cmoe}). After declaration of an \<@OwningCollection> collection, the \<@MustCallOnElements> and \<@CalledMethodsOnElements> type of the collection are both empty. Objects can be added to an \<@OwningCollection> collection with the expected methods, such as \<List.add(E)>. This operation sets the \<@MustCallOnElements> type of the collection to the union of its previous \<@MustCallOnElements> type values and the \<@MustCall> type values of the added element.

The following simple example illustrates how a Socket can be added to a collection. Calling \<add(Socket)> on the \<@OwningCollection> \textit{socketList} removes the obligation for the Socket object and instead unites its \<@MustCall> type values with the \<@MustCallOnElements> type values of \textit{socketList}, meaning in this case that \textit{socketList} now has type  \<@MustCallOnElements({"close"})>.

\begin{verbatim}
@OwningCollection List<Socket> socketList;
socketList = new List<Socket>();
// @MustCallOnElements({}) @CalledMethodsOnElements({})
socketList.add(new Socket(myHost, myPort));
// @MustCallOnElements({"close"}) @CalledMethodsOnElements({})
\end{verbatim}

\subsectionAndLabel{Assignment of resource-holding collections}{resource-leak-collections-assignment}
Similar to \<@OwningCollection> collections, \<@OwningCollection> collection variables may only be assigned to a new instance. A non-\<@OwningCollection> reference may not be assigned to an \<@OwningCollection> reference either. These two rules together directly prevent any intra-procedural aliasing for any \<@OwningCollection> variable and it follows that an \<@OwningCollection> reference is always the only one to the underlying collection within a procedure.

When an \<@OwningCollection> is passed as an argument to a method, the method parameter must also be annotated \<@OwningCollection> for soundness reasons. If it was not, the collection elements with possibly open calling obligations could be overwritten to null, which would not produce a warning (throwing warnings for null assignments of collection/collection elements would produce many false positives for code that is not dealing with resources whatsoever) and thus be unsound.

For such a method call, the calling obligations of the \<@OwningCollection> reference at call-site are passed to the invoked method, which has to fulfill them before returning. The obligations at call-site are thus considered fulfilled.

If the called method is a constructor, it may store the \<@OwningCollection> in a field. Thus, when the constructor returns, the collection reference at call-site may have a field as an alias. The call-site reference thus becomes read-only. For details, see Section~\ref{resource-leak-collections-parameters} and Section~\ref{resource-leak-collections-constructor}.

\subsectionAndLabel{Directly fulfilling \<@MustCallOnElements> obligations on \<@OwningCollection> collections}{resource-leak-collections-fulfilling-obligations}
Analogous to \<@OwningCollection> collections, the \<@MustCallOnElements> obligations of an \<@OwningCollection> collection can be fulfilled with a for-loop or enhanced-for-loop, which iterate over the collection and call methods on each element of the collection. The called methods are determined by checking the \<@CalledMethods> type of the collection element at the end of the loop body, without considering exceptional states, i.e.\ only exception-free executions of the loop body are considered.

For example, the following loop calls the methods \<close()> and \<shutdownOutput()> on the elements of \textit{socketList}.
\begin{verbatim}
// socketList: @MustCallOnElements({"close"}) @CalledMethodsOnElements({})
for (Socket s: socketList) {
    s.shutdownOutput();
    try {
        s.close();
    } catch (Exception e) {}
    // socketList: @MustCallOnElements({"close"}) @CalledMethodsOnElements({})
}
// socketList: @MustCallOnElements({}) @CalledMethodsOnElements({"close", "shutdownOutput"})
\end{verbatim}
The type of the collection is changed only after the loop, not within the loop. The loop is treated as a single instruction with regards to its effect on the type system for this checker.

For such an obligation-fulfilling enhanced-for-loop, the loop body may not contain any break or return statements.

A conventional for-loop may also be used, however, the rules are more strict:
\begin{itemize}
  \item The loop must iterate from 0 to \<Collection.size()>.
  \item The update must be either a pre-increment or post-increment.
  \item When referring to the collection element in the loop body, Collection.get(\<i>) must be used, where i is the loop iterator variable.
  \item The loop body must not contain any return or break statements or writes to the loop variable.
\end{itemize}

For example:
\begin{verbatim}
// socketList: @MustCallOnElements({"close"}) @CalledMethodsOnElements({})
for (int i = 0; i < socketList.size(); i++) {
    try {
        socketList.get(i).close();
    } catch (Exception e) {}
}
// socketList: @MustCallOnElements({}) @CalledMethodsOnElements({"close"})
\end{verbatim}

There are no preconditions for such a loop regarding the \<@MustCallOnElements> type of the collection.

Since the \<@CalledMethods> type of the collection element is checked in the loop body to determine the methods the loop calls on the collection elements, passing the element as an argument to a method annotated with \<@EnsuresCalledMethods> annotation also works:

\begin{verbatim}
// socketList: @MustCallOnElements({"close"}) @CalledMethodsOnElements({})
for (Socket s : socketList) {
    method(s); // method calls 'close' on s
}
// socketList: @MustCallOnElements({}) @CalledMethodsOnElements({"close"})

...

@EnsuresCalledMethods(value="#1", methods={"close"})
void method(Socket s) {
    try {
        s.close();
    } catch (Exception e) {}
}

\end{verbatim}
\sectionAndLabel{Collections of resources (old)}{resource-leak-collections-old}

The Resource Leak Checker cannot verify code that stores a collection of
resources in a generic collection (e.g., \<java.util.List>) and then
resolves the obligations of each element of the collection at once (e.g.,
by iterating over the \<List>).  In the future, the checker will support
this.  The remainder of this section explains the implementation issues;
most users can skip it.

The first implementation issue is that \<@Owning> and \<@NotOwning> are
declaration annotations rather than type qualifiers, so they cannot be
written on type arguments. It is possible under the current design to have
an \code{@Owning List<Socket>}, but not a \code{List<@Owning Socket>}.
It would be better to make \<@Owning> a type annotation, but this is a
challenging design problem.

The second implementation issue is the defaulting rule for \<@MustCall> on
type variable upper bounds.  Currently, this default is \<@MustCall(\{\})>,
which prevents many false positives in code with type variables that makes
no use of resources --- an important design principle.
However, this defaulting rule does have an unfortunate consequence: it is
an error to write \code{List<Socket>} or any other type with a concrete
type argument where the type argument itself isn't \<@MustCall(\{\})>. A programmer who
needs to write such a type while using the Resource Leak Checker has a few
choices, all of which have some downsides:

\begin{itemize}
\item Write \code{List<? extends Socket>}. This rejects calls to \<add()>
or other methods that require an instance of the type variable, but it
preserves some of the behavior (e.g., calls to \<remove()> are permitted).
This is the best choice most of the time if the \<List> is not intended to
be owning.
\item Write \code{List<@MustCall Socket>}. This makes it an error to
add a Socket to the list, since the type of the Socket is
\<@MustCall("close")> but the list requires \<@MustCall()>.
\item Suppress one or more warnings.
\end{itemize}

The recommended way to use the Resource Leak Checker in this situation is
to rewrite the code to avoid a \<List> of owning resources. If rewriting is
not possible, the programmer will probably need to suppress a warning and
then verify the code using a method other than the Resource Leak Checker.


\sectionAndLabel{Resource Leak Checker annotation inference algorithm}{resource-leak-checker-inference-algo}

The Resource Leak Checker uses a specialized algorithm to infer annotations
when whole program inference (WPI, Section~\ref{whole-program-inference})
is enabled.  The algorithm is described in the paper ``Inference of
Resource Management Specifications''~\cite{ShadabGTEKLLS2023} (OOPSLA 2023,
\myurl{https://homes.cs.washington.edu/~mernst/pubs/resource-inference-oopsla2023-abstract.html}).


% LocalWords:  de subchecker OutputStream MustCall MustCallUnknown RAII Un
% LocalWords:  PolyMustCall InheritableMustCall MultiRandSelector callsite
% LocalWords:  Partitioner CalledMethods AnoLightweightOwnership NotOwning
% LocalWords:  AnoResourceAliasing MustCallAlias AnoCreatesMustCallFor
% LocalWords:  EnsuresCalledMethods CreatesMustCallFor expr Verification''
% LocalWords:  closeSocket destructor'' destructor BufferedOutputStream
% LocalWords:  AnoResourceAliases createsmustcallfor SocketContainer
% LocalWords:  CalledMethodsBottom CollectionIndexOutOfBoundsException
% LocalWords:  NegativeCollectionSizeException
